\newglossaryentry{glos:rest-api}{
    name={REST-Schnittstelle},
    plural={REST-Schnittstellen},
    description={
        Eine \emph{REST-Schnittstelle} bezeichnet eine Schnittstelle, welche über einen \gls{glos:webserver} zugänglich ist und dem \emph{Representational State Transfer}-Architekturstil folgt.
        Dieser Stil beschreibt die Entwicklung von verteilten Systemen durch zustandslose Operationen, d.\ h.\ der Klient kann diese in jeder beliebigen Reihenfolge anfragen}
}

\newglossaryentry{glos:webserver}{
    name={Webserver},
    plural={Webserver},
    description={
        Unter \emph{Webserver} versteht man eine Software, welche Anwendungen und Webseiten über HTTP zur Verfügung stellt}
}

\newglossaryentry{glos:geschaeftsprozess}{
    name={Geschäftsprozess},
    plural={Geschäftsprozesse},
    description={
        Als \emph{Geschäftsprozess} wird eine Folge von wertschöpfenden Aktivitäten bezeichnet, wobei eine beliebige Anzahl von Eingaben eine klar definierte kundenorientierte Ausgabe erzeugt.
        Nur sich wiederholende Abläufe werden als Geschäftsprozess bezeichnet}
}

\newglossaryentry{glos:geschaeftsprozessmodell}{
    name={Geschäftsprozessmodell},
    plural={Geschäftsprozessmodelle},
    description={
        \emph{Geschäftsprozessmodelle} sind graphische, formale oder mathematisch Abbildungen eines realen Prozessablaufs und können in unterschiedlichen Abstraktionsebenen und variierendem Detailgrad vorliegen}
}

\newglossaryentry{glos:geschaeftsprozess-ausfuehrbar}{
    name={Ausführbarer Geschäftsprozess},
    plural={Ausführbare Geschäftsprozesse},
    description={
        Ein \emph{ausführbares Geschäftsprozessmodell} beschreibt ein \Gls{glos:geschaeftsprozessmodell} mit einem sehr hohen Grad an Detailinformationen, welches in einer formalen Beschreibungssprache vorliegt.
        Es umfasst zumeist eine Vielzahl an automatisierbaren Aktivitäten und wird um etliche technische Informationen, wie etwa Bedingungen für einzelne Pfade des Ablaufs, ergänzt.
        Dadurch wird eine computergestützte Verarbeitung der beschriebenen Aktivitätsfolge ermöglicht}
}

\newglossaryentry{glos:prozessengine}{
    name={Prozessengine},
    plural={Prozessengine},
    description={
        Mit \emph{Prozessengine} wird eine Software bezeichnet, die \glsdisp{glos:geschaeftsprozess-ausfuehrbar}{ausführbare Geschäftsprozessmodelle} verarbeiten kann.
        Dafür ist sie in der Lage die formal definierten Aktivitäten in korrekter zeitlicher Abfolge und unter Einhaltung etwaiger Bedingungen abzuarbeiten.
        Pro Ausführung eines Modells wird eine neues \gls{glos:prozessexemplar} erzeugt}
}

\newglossaryentry{glos:prozessexemplar}{
    name={Prozessexemplar},
    plural={Prozessexemplare},
    description={
        Ein \emph{Prozessexemplar}, auch als Prozessinstanz bezeichnet, ist ein \gls{glos:geschaeftsprozess} während seiner Ausführung.
        Es umfasst einen, an eine \gls{glos:geschaeftsprozessmodell} gebundenen, Kontext, welcher Informationen über den aktuellen Abarbeitungsstand des Prozesses enthält}
}

\newglossaryentry{glos:debugging}{
    name={Debugging},
    description={
        \emph{Debugging} bezeichnet eine Technik in der Angewandten Informatik, die es ermöglicht während des Softwareentwicklungsprozesses Fehlerpunkte in einem Programm aufzuspüren.
        Dazu markiert der Entwickler beliebige Stellen im \gls{glos:quellcode} der Software, sogenannte \glspl{glos:debugger-haltepunkt}.
        Während der Ausführung der Software wird, sobald einer dieser Haltepunkte erreicht ist, die Programmabwicklung unterbrochen und eine Oberfläche zur Verfügung gestellt, um das weitere Vorgehen zu bestimmen.
        Aus dieser Sicht heraus werden zusätzliche Funktionalitäten angeboten, um den internen Programmzustand zu analysieren}
}

\newglossaryentry{glos:quellcode}{
    name={Quellcode},
    description={
        \emph{Quellcode} ist eine menschenles- und für Computer übersetzbare Abfolge von Programmanweisungen, die zusammen ausgeführt eine bestimmte Funktionalität auf dem Computer bereitstellen}
}

\newglossaryentry{glos:debugger-unterbrochen-prozessexemplar}{
    name={Unterbrochenes Prozessexemplar},
    plural={Unterbrochene Prozessexemplare},
    description={
        Ein unter Kontrolle durch den \gls{glos:debugger} stehendes Prozessexemplar ist pausiert.
        Es wird als pausiertes oder \emph{unterbrochenes Prozessexemplar} bezeichnet}
}

\newglossaryentry{glos:debugger}{
    name={Debugger},
    plural={Debugger},
    description={
        \glsdisp{glos:ide}{Moderne Entwicklungsumgebungen} stellen, um den Prozess des Debugging zu unterstützen, verschiedene Werkzeuge und Funktionen zur Verfügung.
        Dieser Werkzeugverbund wird als \emph{Debugger} bezeichnet}
}

\newglossaryentry{glos:debugger-haltepunkt}{
    name={Haltepunkt},
    plural={Haltepunkte},
    description={
        Ein \emph{Haltepunkt} ist eine temporär definierte Stelle im Ablaufcode eines Programms.
        Diese zeigt dem Debugger an, dass der Programmablauf dort für die Analyse unterbrochen werden soll}
}

\newglossaryentry{glos:ide}{
    name={Integrierten Entwicklungsumgebung},
    plural={Integrierten Entwicklungsumgebungen},
    description={
        Bei einer \emph{Integrierten Entwicklungsumgebung}, kurz IDE, handelt es sich um einen Werkzeugkasten, welcher alle zur Softwareentwicklung benötigten Komponenten unter einer Oberfläche zusammenfasst}
}

\newglossaryentry{glos:thread}{
    name={Thread},
    plural={Threads},
    description={
        \devnote{Tread}}
}
